\documentclass[../master.tex]{subfiles}
\usepackage{amsmath}
\usepackage{amsfonts}

\begin{document}

    \section{Bleichen bacher oracle attack}
    \subsection{overview of the attack}
    Suppose we have access to an Oracle which can tell us if an arbitrary cipher text is PKCS conforming and got hold
    of the ciphertext c of the message m.
    The attacker would like to find
    \begin{align*}
        m \equiv c^d \bmod n
    \end{align*}
    The attacker choose a random s and calculates
    \begin{align*}
        c' \equiv c \cdot s^e \bmod n && s \in \mathbb{N}
    \end{align*}
    Which can then be send to the oracle asking if it's PKCS conforming.
    Remember from RSA that given the public key (e, n) and private key (d, n) it holds that
    \begin{align*}
        c' \equiv c \cdot s^e \bmod n
    \end{align*}
    and that the oracle will try to decrypt the cipher and return whether that's PKCS conforming.
    \begin{align*}
    (c')^d \equiv (c \cdot s^e)^d \bmod n \ \ \Rightarrow \ \ (c')^d \equiv m \cdot s \bmod n
    \end{align*}
    Therefore if the oracle says that the ciphertext c' is PKCS conforming
    we'll know that the first 2 bytes of $m \cdot s$ is 00 and 02.
    We have then extracted a tiny bit of information from the message by asking the oracle.
    \begin{align*}
        2B \leq m \cdot s \bmod n < 3B
    \end{align*}
    Where $B = 2^{8(k-2)}$ and k is the length of n in bytes.
    Repeating this for different integers s and using previous obtained information,
    we'll eventually narrow the solution so much that we can dervice m.

    % question: Why is this value 2B \leq ms < 3B, where does the number 2 and 3 come from?

    \subsection{The algorihtm}


\end{document}